\documentclass[12pt, a4paper,oneside]{book}
\newcommand{\re}{\mathrm{e}}
\newcommand{\ri}{\mathrm{i}}
\newcommand{\rd}{\mathrm{d}}
\def\semicolon{\nobreak\mskip2mu\mathpunct{}\nonscript\mkern-\thinmuskip{;}
\mskip6muplus1mu\relax} % This defines the semicolon command

        % allows index generation
\usepackage{graphicx}        % standard LaTeX graphics tool
                             % when including figure files
\usepackage{multicol}        % used for the two-column index




\usepackage{color,tikz}
%\usepackage[unicode,bookmarks,bookmarksopen,bookmarksopenlevel=2,colorlinks,linkcolor=blue,citecolor=green]{hyperref}

\usepackage{amsmath,eucal,amssymb}
\usepackage{mathrsfs,graphicx,texdraw}
\usepackage{fancyhdr,framed}
\usepackage{tikz, tikz-3dplot}
\usepackage{tkz-euclide}
\usetikzlibrary{decorations.fractals}
\usetikzlibrary{decorations.footprints}




\usepackage{palatino}

\usepackage[latin1]{inputenc}

\usepackage[T1]{fontenc}
%\usepackage[dvips]{graphicx}
%\usepackage{times}


\definecolor{prempurple}{HTML}{37003c} % purple
\definecolor{premgreen}{HTML}{00ff87} % green
\definecolor{prempink}{HTML}{ff2882} % pink


\def\grole{\mathrel{\mathchoice {\vcenter{\offinterlineskip\halign{\hfil
$\displaystyle##$\hfil\cr>\cr\noalign{\vskip-1.5pt}<\cr}}}
{\vcenter{\offinterlineskip\halign{\hfil$\textstyle##$\hfil\cr
>\cr\noalign{\vskip-1.5pt}<\cr}}}
{\vcenter{\offinterlineskip\halign{\hfil$\scriptstyle##$\hfil\cr
>\cr\noalign{\vskip-1pt}<\cr}}}
{\vcenter{\offinterlineskip\halign{\hfil$\scriptscriptstyle##$\hfil\cr
>\cr\noalign{\vskip-0.5pt}<\cr}}}}}

\newenvironment{rcases}
  {\left.\begin{aligned}}
  {\end{aligned}\right\rbrace}

\newenvironment{lcases}
  {\left\lbrace\begin{aligned}}
  {\end{aligned}\right.}

\newtheorem{theorem}{Theorem}[section]
\newtheorem{lemma}[theorem]{Lemma}
\newtheorem{proposition}[theorem]{Proposition}
\newtheorem{corollary}[theorem]{Corollary}
\newtheorem{definition}[theorem]{Definition}
\newtheorem{example}[theorem]{Example}
\newtheorem{remark}[theorem]{Remark}

\newenvironment{proof}[1][Proof]{\begin{trivlist}
\item[\hskip \labelsep {\bfseries #1}]}{\end{trivlist}}
\newenvironment{solution}[1][Solution]{\begin{trivlist}
\item[\hskip \labelsep {\bfseries #1}]}{\end{trivlist}}

\newcommand{\qed}{\nobreak \ifvmode \relax \else
      \ifdim\lastskip<1.5em \hskip-\lastskip
      \hskip1.5em plus0em minus0.5em \fi \nobreak
      \vrule height0.75em width0.5em depth0.25em\fi}

\numberwithin{equation}{section}
\def\Ad{{\mbox{Ad}}}
\def\im{{\mbox{Im}}}
\def\Re{{\mbox{Re}\;}}
\def\ad{\mathrm{ad\,}}
\def\openone{\leavevmode\hbox{\small1\kern-3.3pt\normalsize1}}
\def\Res{\mathop{\mbox{Res}\,}\limits}
\def\biglb{\big[\hspace*{-.7mm}\big[}
\def\bigrb{\big]\hspace*{-.7mm}\big]}


\def\bigrbt{\mathop{\bigrb }\limits_{\widetilde{\;}}}
\def\biggrbt{\mathop{\biggrb }\limits_{\widetilde{\;}}}
\def\Bigrbt{\mathop{\Bigrb }\limits_{\widetilde{\;}}}
\def\Biggrbt{\mathop{\Biggrb }\limits_{\widetilde{\;}}}

\def\bPhi{\mathbf{\Phi}}
\def\bM{\mathbf{M}}
\def\bm{\mathbf{m}}
\def\bbbc{{\Bbb C}}
\def\bbbr{{\Bbb R}}
\def\bbbz{{\Bbb Z}}
\def\bbbs{{\Bbb S}}
\def\diag{\mbox{diag}\,}
\def\tr{\mbox{tr}\,}

\textwidth=17cm   \textheight=24.5cm \voffset=-2cm

\evensidemargin=-0.5cm \oddsidemargin=-0.5cm

 \renewcommand{\baselinestretch}{1.5}

%%%%%%%%fancy header%%%%%%%%%%%%%%%%%

\usepackage{fancyhdr}
\pagestyle{fancy}
\usepackage{calc}
\newlength{\pageoffset}
\setlength{\pageoffset}{0cm} % use whatever you like
\fancyheadoffset[LE,RO]{\pageoffset}
\renewcommand{\chaptermark}[1]{\markboth{#1}{}}
\renewcommand{\sectionmark}[1]{\markright{\thesection\ #1}}
\fancyhf{}
\fancyhead[LE]{\makebox[\pageoffset][l]{\thepage}\hfill\leftmark}
\fancyhead[RO]{\rightmark\hfill\makebox[\pageoffset][r]{\thepage}}
\fancypagestyle{plain}{%
    \fancyhead{} % get rid of headers
    \renewcommand{\headrulewidth}{0pt} % and the line
}

%%%%%%%%%%%%%%%%%%%%%%%%%%%%%%%%%%%%

\arraycolsep=2pt

%%%%Fancy Chapter%%%

%\usepackage{lmodern}
\usepackage{graphicx}
\usepackage{xcolor}
\usepackage{titlesec}
\usepackage{microtype}
\usepackage{lipsum}

\titleformat{\chapter}[display]
  {\normalfont\bfseries\color{prempurple}}
  {\filleft\hspace*{-60pt}%
    \rotatebox[origin=c]{90}{%
      \normalfont\color{prempurple}\Large%
        \textls[180]{\textsc{\chaptertitlename}}%
    }\hspace{10pt}%
    {\setlength\fboxsep{0pt}%
    \colorbox{prempurple}{\parbox[c][3cm][c]{2.5cm}{%
      \centering\color{premgreen}\fontsize{80}{90}\selectfont\thechapter}%
    }}%
  }
  {10pt}
  {\color{premgreen}\titlerule[2.5pt]\vskip3pt
  \titlerule\vskip10pt\color{prempurple}\LARGE\sffamily}

% hyperref should be loaded after all other packages
% aliascnt is needed to get \autoref (from hyperref) to work correctly with custom amsthm theorems
\usepackage{aliascnt}
\usepackage[colorlinks,linkcolor=prempink,citecolor=premgreen, linktoc=all]{hyperref}

% removes dots in table of contents
\makeatletter
\renewcommand\@dotsep{140}   % default value 4.5
\makeatother

\usepackage{advdate}

\newcommand{\yesterday}{{\AdvanceDate[-1]\today}}

\newcommand{\tomorrow}{{\AdvanceDate[1]\today}}


\begin{document}


\thispagestyle{empty}

\begin{minipage}{0.2\textwidth}
\centerline{\includegraphics[width=.4\textwidth]{essex} }
\end{minipage}
\begin{minipage}{0.8\textwidth}

$ \qquad \qquad \qquad ${\LARGE \bf \sl University of Essex}

{\LARGE \bf Department of Mathematical Sciences}

\end{minipage}

\begin{center}

\noindent\textcolor{prempurple}{\rule{\linewidth}{4.8pt}}

\vspace{1cm}

{\LARGE \sc  MA838: Capstone Project}

\vspace{1.5cm}

{\Huge{A Data Analytics Approach to Fantasy Football Management}}

\vspace{1.5cm}

{\Large \bf Reece Lance}

\vspace{0.5cm}

{\Large \bf 1804752}

\vspace{1.5cm}

\centerline{\includegraphics[width=0.75\textwidth]{prem_logo.jpeg}}

\vspace{1.5cm}

{\Large {Supervisor:} {\bf Dr Andrew Harrison}}

\vspace{.25cm}

\noindent\textcolor{prempurple}{\rule{\linewidth}{4.8pt}}

\vspace{1cm}
{\Large \today }\\[4pt]
\end{center}
\newpage

\chapter{Project Outline}\label{ch:1}

For the following year, I intend to study `A Data Analytics Approach to Fantasy Football
Management'.

\section{Description}\label{sec:1.1}

This project will involve extensive research into the analytical side of football management,
proving and disproving theories about the fantasy football game. I will use data analysis
skills to collect, clean and interpret data from the Premier League, resulting in accurate
predictions of the best teams and tactics to use throughout the game's course. The final
product will be an application, with the main functionality of presenting the predictions in an
effective way for the user to follow. The application will also include other features such as
showing live scores, league tables and player rating systems, similar to the Premier League
official website and phone application. It will also be made possible to sign into your Fantasy
Premier League (FPL) account, to see your current team and any past statistics and history.
The Capstone will consist of full documentation of my research findings, processes, data
analysis, predictions and application creation and features.

\section{Process}\label{sec:1.2}

Following the steps of basic data analysis, I will begin by retrieving the raw data in the JSON
format. Next, I will store and manage the data in Data Frames, splitting the data up where
necessary. Then I will begin cleaning the data by ensuring there are no missing values, and
that the data is in the correct data types. The data will then be analysed, and new data created
in order to improve the predictions. Furthermore, the game rules will then be addressed,
as the predictions must follow all of the rules and the scoring system to be effective. These
predictions must then be integrated into Java to be used in the application. Moreover, in Java
I will produce the application to present the results, including any additional features. Also
in Java, I will create a login page in order for users with an FPL account to have access to their
data and statistics. I will ensure that the account details are secure. To prove the success of
the prediction and application, I will have users test the application and assess some markers
(these will be decided later). To test the prediction accuracy, we can test the program from
the start of the season and assess how accurate the predictions are.

\section{Languages and Libraries}\label{sec:1.3}

The back-end of the program will be written in Python, using some external libraries; the
front-end of the application will be written in Java. The data that will be used is from the
Fantasy Premier League API and will be retieved using `requests', a Python HTTP library.
When the data is retrieved, a local version of the data will be saved or updated for when
the API cannot be accessed; this will be in a JSON file. To clean and manipulate the data I
will use `Pandas', and for the analysis, `matplotlib'. The application's GUI (Graphical User
Interface) will be written using the Java library `Swing', with the help of the `Swing Form
Designer' in `IntelliJ IDEA'.

\end{document}